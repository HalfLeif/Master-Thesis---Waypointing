\section{Method}

\subsection{Network architecture}

\subsubsection{Experiments}
% TODO: write here or somewhere completely different?

\subsubsection{Encoder}

\begin{figure}
\centering

\begin{tikzpicture}
% (1, 900, 1500, 1)
% (1, 450, 750, 4)
% (1, 225, 375, 16)
% (1, 225, 375, 32)
% (1, 112, 187, 32)
% (1, 112, 187, 128)
% (1, 56, 93, 128)
% (1, 56, 89, 256)
% (1, 28, 44, 256)

%   \pic {annotated cuboid};
  \pic [fill=cyan, draw=blue] at (0,0) {annotated cuboid={width=32, height=900, depth=1500}};
  \pic [fill=cyan, draw=blue] at (0.5,0) {annotated cuboid={width=32, height=900, depth=1500}};

  \pic [fill=yellow, draw=white] at (1.5,-0.5) {annotated cuboid={width=32, height=450, depth=750}};
  \pic [fill=cyan, draw=blue] at (2.1,-0.5) {annotated cuboid={width=64, height=450, depth=750}};
  \pic [fill=cyan, draw=blue] at (2.7,-0.5) {annotated cuboid={width=64, height=450, depth=750}};

  \pic [fill=yellow, draw=white] at (3.5,-0.7) {annotated cuboid={width=64, height=225, depth=375}};
  \pic [fill=cyan, draw=blue] at (4.5,-0.7) {annotated cuboid={width=128, height=225, depth=375}};

  \pic [fill=yellow, draw=white] at (5.5,-0.7) {annotated cuboid={width=128, height=112, depth=187}};
  \pic [fill=cyan, draw=blue] at (6.5,-0.7) {annotated cuboid={width=128, height=112, depth=187}};

  \pic [fill=yellow, draw=white] at (7.5,-0.8) {annotated cuboid={width=128, height=56, depth=93}};
  \pic [fill=green, draw=blue] at (9,-0.8) {annotated cuboid={width=256, height=56, depth=89}};

  \pic [fill=yellow, draw=white] at (10.5,-0.8) {annotated cuboid={width=256, height=28, depth=44}};

%   \pic [fill=magenta, text=blue, draw=blue] at (5,0) {annotated cuboid={width=2, height=50, depth=30}};
%   \pic [fill=green, text=green!50!black, draw=green!25!black] at (5,-2) {annotated cuboid={width=6, height=20, depth=15, units=mm}};
%   \pic at (1,-3) {annotated cuboid={width=150, height=200, depth=250, scale=.01, units=m}};
%   \pic [fill=cyan, text=blue!75!cyan, draw=blue!75!cyan] at (-3,-2) {annotated cuboid={width=15, height=18, depth=13.5, units=}};
\end{tikzpicture}

\caption{Proposed architecture for the encoder. Convolutional layers are blue (3x3) and green (1x5) while max pool layers are yellow (2x2).
The number below each layer indicate the depth of the layer.}
\label{fig:encoder}
\end{figure}


For encoder, we use a CNN with 7 convolutional layers with additional pooling layers, see figure \ref{fig:encoder}.
% We experimented with several different variations in number of convolution layers, pooling layers and layer depths before choosing this model.
Two consecutive 3x3 convolutional layers is a refactoring of a single 5x5 layer as suggested by \cite{InceptionV3}. Thus, the first four convolutional layers correspond to two 5x5 convolutional layers with pooling, similar to the original single digit CNN classifier \cite{lecun_1989} but with greater depth.

The number of pooling layers, number of convolutional layers and depth per layer were determined experimentally. The final encoder design, as in figure \ref{fig:encoder}, is somewhat similar to the encoder used in \cite{FornesCnnCategorization} except that we use more aggressive pooling. This is necessary because in our work, we input images of entire pages while \cite{FornesCnnCategorization} classify word images which presumably are much smaller.

%The first four layers correspond closely to the original single digit classifier CNN \cite{lecun_1989} with the exception

% The number of layers and their depth was determined experimentally on the synthesized dataset using previous work as starting point \cite{FornesCnnCategorization}.

% TODO describe experiments.



\subsubsection{Attention model}

Another major difference between our work and \cite{FornesCnnCategorization} is that although we both use variable input size, they use \textbf{Spatial pyramid pooling} while we apply a soft attention model as presented in \cite{AttendAndTell}. Like we discussed previously in \ref{ssec:attention}, the encoder output can be seen as a list of feature vectors $\matbf{a_i}$ which correspond to different but overlapping locations.

The attention model consists of a multi layer perceptron which for each feature vector computes a salience score $e_i$:

\[
e_i = f_\text{MLP}(\mathbf{a_i})
\]

Note that this MLP does not have access to any information about the location of $a_i$, thus the network must learn to recognize important parts by looking at the encoded features alone instead of learning to always look at a certain position in the input image. This is especially important for population records where the relevant information can be in different locations depending on scribe and time period.

As discussed, we normalize the $e_i$ into $\alpha_i$ by computing the softmax. Then we use the attentions ${\alpha_i}$ as weights over the feature vectors:

\[
\mathbf{z} = \sum_{k=1}^L \alpha_i \mathbf{a_i}
\]

Note that the spatial information about where in the image $\mathbf{z}$ comes from is lost in this sum. So just like the attention MLP, the decoder also becomes invariant to the object's position in the input image.

\subsubsection{Decoder}



\subsection{Pre-training}
\subsection{Labeling}
\subsection{Training}
\subsection{Evaluation}


% TODO should technologies/implementation be described in a different section?
% Is it at all relevant to discuss about Tensorflow?
\subsection{Tensorflow}
% TODO Mention GPU acceleration here or in section about neural networks?

\subsection{Hardware}
% TODO Mention hardware and training time .
