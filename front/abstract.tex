
Waypointing of Historical Documents by Deep Learning\\
% A Subtitle that can be Very Much Longer if Necessary\\
LEIF SCHELIN\\
Department of Computer Science and Engineering\\
Chalmers University of Technology and University of Gothenburg\setlength{\parskip}{0.5cm}

\thispagestyle{plain}			% Supress header
% \setlength{\parskip}{0pt plus 1.0pt}
\section*{Abstract}

We introduce a new dataset for extracting genealogical information consisting of Swedish population records and weak labels.
Inspired by recent progress in deep learning and computer vision, we suggest and evaluate an end-to-end trainable deep learning model based on convolutional neural networks as well as several loss functions for extracting the year from each page.
We show how our model partially recognizes year from locating and transcribing the written year and partially from recognizing typical handwriting styles.
Finally we provide a baseline on this dataset and suggest future work for automatically extracting genealogical information from population records.

% KEYWORDS (MAXIMUM 10 WORDS)
\vfill
Keywords: Deep learning, convolutional neural networks, attention models, handwriting recognition, information extraction, genealogy.

\newpage				% Create empty back of side
\thispagestyle{empty}
\mbox{}
