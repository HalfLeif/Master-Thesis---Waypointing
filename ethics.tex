
\section{Genealogy}

Genealogy is the study of one's ancestors. It involves finding who they were, the time and place of their birth, where they lived, how their life was, how their life eventually ended and how their family continued to present day.

Getting to know and comprehend the lives of one's ancestors can have great personal significance as one can feel more connected with people in centuries past. One can gain appreciation for parents and forefathers and their efforts so that oneself could be born.
Experiences can also be learned from important events, stories and hardships in the lives of one's ancestors and one can more easily appreciate how many things are different today, although as human beings we are perhaps not so different.

Many people are familiar with the lives of their parents and perhaps also the their grandparents from just listening to them.
However, each generation further back requires more study and effort to explore as the records get more sparse, get harder to access, are written in an older language and in a handwriting style that get increasingly different from the handwriting of today. Furthermore, because each child has two biological parents, the number of ancestors increases at least by a factor of two in each generation.

The advance of technology has been very instrumental in making the genealogical research easier and more accessible. Instead of traveling to different vaults of the records, people could order microfilms with photographs to view at home or at a nearby genealogy center. With the Internet, people can currently access many records directly online as images. A future big leap forward consists of the indexing effort which makes it possible for people to search the records with search engines and databases instead of manually scanning through and reading the pages.

Manually indexing a book from cover to cover has a very high throughput of data compared to searching the pages for a single entry because 1. it takes time to get accustomed to the handwriting of that particular scribe as well as locations in that parish 2. the time for each entry is significantly reduced as one does not need to search through many pages. So although it is not viable for a single person to do, if many people contribute, it lowers the total workload in the end for the community.

However, indexing takes a lot of effort and also some experience in reading old handwriting. If computers could index the records themselves with only minor human contribution, that would significantly improve the indexing rate and hence produce a very powerful tool for the entire genealogical community.

Solving waypointing is not as significant as indexing but would constitute a stepping stone towards indexing in extracting information from the records in an automated manner.
% as well as reducing manual labor for waypointing.
Waypointing in itself is also useful as more images can be published online under the correct categories. Automating waypointing would free time in favor of genealogical research and indexing.
