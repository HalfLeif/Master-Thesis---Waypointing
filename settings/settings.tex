
% BASIC SETTINGS
\usepackage{moreverb}								% List settings
\usepackage{textcomp}								% Fonts, symbols etc.
\usepackage{lmodern}								% Latin modern font
\usepackage{helvet}									% Enables font switching
\usepackage[T1]{fontenc}							% Output settings
\usepackage[english]{babel}							% Language settings
\usepackage[utf8]{inputenc}							% Input settings
\usepackage{amsmath}								% Mathematical expressions (American mathematical society)
% \usepackage{amssymb}								% Mathematical symbols (American mathematical society)
\usepackage{graphicx}								% Figures
% \usepackage{subfig}									% Enables subfigures
\numberwithin{equation}{chapter}					% Numbering order for equations
\numberwithin{figure}{chapter}						% Numbering order for figures
\numberwithin{table}{chapter}						% Numbering order for tables
\usepackage{minted}						    		% Enables source code listings
% \usepackage{chemfig}								% Chemical structures
\usepackage[top=3cm, bottom=3cm,
			inner=3cm, outer=3cm]{geometry}			% Page margin lengths			
\usepackage{eso-pic}								% Create cover page background
\newcommand{\backgroundpic}[3]{
	\put(#1,#2){
	\parbox[b][\paperheight]{\paperwidth}{
	\centering
	\includegraphics[width=\paperwidth,height=\paperheight,keepaspectratio]{#3}}}}
\usepackage{float} 									% Enables object position enforcement using [H]
\usepackage{parskip}								% Enables vertical spaces correctly 


\usepackage[utf8]{inputenc}
\usepackage[english]{babel}
\usepackage{csquotes}
\usepackage{hyperref} % For hyperlinks in the PDF
\usepackage{url}
\usepackage{amsfonts}
\usepackage{tikz}
\usepackage{subcaption}

\usepackage{footnote}
\makesavenoteenv{tabular} % For footnotes inside tabular
\makesavenoteenv{table}

\usepackage[toc,page]{appendix}

%\usepackage[backend=biber]{biblatex}
\usepackage[style=authoryear-icomp,
            urldate=iso8601,
            maxbibnames=99,
            maxcitenames=3,
            backend=biber]{biblatex}


% OPTIONAL SETTINGS (DELETE OR COMMENT TO SUPRESS)

% Disable automatic indentation (equal to using \noindent)
\setlength{\parindent}{0cm}                         


% Caption settings (aligned left with bold name)
\usepackage[labelfont=bf, textfont=normal,
			justification=justified,
			singlelinecheck=false]{caption} 		

		  	
% Activate clickable links in table of contents  	
\usepackage{hyperref}								
\hypersetup{colorlinks, citecolor=black,
   		 	filecolor=black, linkcolor=black,
    		urlcolor=black}


% Define the number of section levels to be included in the t.o.c. and numbered	(3 is default)	
\setcounter{tocdepth}{5}							
\setcounter{secnumdepth}{5}	


% Chapter title settings
\usepackage{titlesec}		
\titleformat{\chapter}[display]
  {\Huge\bfseries\filcenter}
  {{\fontsize{50pt}{1em}\vspace{-4.2ex}\selectfont \textnormal{\thechapter}}}{1ex}{}[]


% Header and footer settings (Select TWOSIDE or ONESIDE layout below)
\usepackage{fancyhdr}								
\pagestyle{fancy}  
\renewcommand{\chaptermark}[1]{\markboth{\thechapter.\space#1}{}} 


% Select one-sided (1) or two-sided (2) page numbering
\def\layout{2}	% Choose 1 for one-sided or 2 for two-sided layout
% Conditional expression based on the layout choice
\ifnum\layout=2	% Two-sided
    \fancyhf{}			 						
	\fancyhead[LE,RO]{\nouppercase{ \leftmark}}
	\fancyfoot[LE,RO]{\thepage}
	\fancypagestyle{plain}{			% Redefine the plain page style
	\fancyhf{}
	\renewcommand{\headrulewidth}{0pt} 		
	\fancyfoot[LE,RO]{\thepage}}	
\else			% One-sided  	
  	\fancyhf{}					
	\fancyhead[C]{\nouppercase{ \leftmark}}
	\fancyfoot[C]{\thepage}
\fi


% Enable To-do notes
\usepackage[textsize=tiny]{todonotes}   % Include the option "disable" to hide all notes
\setlength{\marginparwidth}{2.5cm} 


% Supress warning from Texmaker about headheight
\setlength{\headheight}{15pt}		



% http://tex.stackexchange.com/questions/285578/how-to-draw-parallelepiped-and-cube-with-latex
\usetikzlibrary{quotes,arrows.meta,arrows}
\tikzset{
  annotated cuboid/.pic={
    \tikzset{%
      every edge quotes/.append style={midway, auto},
      /cuboid/.cd,
      #1
    }
    \draw [every edge/.append style={pic actions, densely dashed, opacity=.5}, pic actions]
    (0,0,0) coordinate (o) -- ++(-\cubescale*\cubex,0,0) coordinate (a) -- ++(0,-\cubescale*\cubey,0) coordinate (b) edge coordinate [pos=1] (g) ++(0,0,-\cubescale*\cubez)  -- ++(\cubescale*\cubex,0,0) coordinate (c) -- cycle
    (o) -- ++(0,0,-\cubescale*\cubez) coordinate (d) -- ++(0,-\cubescale*\cubey,0) coordinate (e) edge (g) -- (c) -- cycle
    (o) -- (a) -- ++(0,0,-\cubescale*\cubez) coordinate (f) edge (g) -- (d) -- cycle;
    \path [every edge/.append style={pic actions, |-|}]
    (b) +(0,-5pt) coordinate (b1) edge ["\cubex \cubeunits"'] (b1 -| c)
    % (b) +(-5pt,0) coordinate (b2) edge ["\cubey \cubeunits"] (b2 |- a)
    % (c) +(3.5pt,-3.5pt) coordinate (c2) edge ["\cubez \cubeunits"'] ([xshift=3.5pt,yshift=-3.5pt]e)
    ;
  },
  /cuboid/.search also={/tikz},
  /cuboid/.cd,
  width/.store in=\cubex,
  height/.store in=\cubey,
  depth/.store in=\cubez,
  units/.store in=\cubeunits,
  scale/.store in=\cubescale,
  width=10,
  height=10,
  depth=10,
  units=,
  scale=.005,
}

\renewcommand{\cite}{\parencite}