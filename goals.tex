\section{Goals}

% The thesis aspires to explore

The goal of the thesis is to make an open-source prototype that automatically extracts high-level information for each photographed page of historical population records. The information under consideration is 1. year, 2. page number and 3. record type, where the record type is one of \{blank, birth, death, marriage\}. Preferably, the system should also suggest which part of the image that contains the year and page number for example by providing a bounding box.
% If time permits, extracting location (such as parish) will also be considered.

The system should learn from the currently indexed and waypointed data in a fully autonomous manner without explicit feature engineering.
% However, standard pre-processing techniques such as median filtering and gray level normalization may be necessary.

The trained model should have as high precision as possible although human precision is unrealistic. The system should also be fast enough that annotating millions of images should be viable within reasonable time on a high-end computer.

%A perfect classification is unrealistic so the resulting groupings of images will need some manual inspection before publication. However, it should be sufficient to check some of the pages in a group instead of going through every page. Thus the manual workload should be significantly decreased. If the precision of the system is too low, correcting errors would be equally troublesome as doing the waypointing manually. Thus, the precision of the classification should be as high as possible.

% TODO verify that this promise is fulfilled!
We intend to experiment with how well learning from one dataset generalizes to other datasets, for example other regions.

Even if the trained model does not achieve wanted precision, the thesis is still valuable as an exploration of different techniques for approaching the long-term goal of automated indexing.
