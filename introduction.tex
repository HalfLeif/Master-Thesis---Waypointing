
\section{Introduction}

FamilySearch is a large non-profit genealogy organization \cite{FamilySearchAbout}. They have been photographing handwritten historical population records in many countries for many years.

In 2006, the \textbf{Indexing} project started where volunteers manually extract relevant information from the photographed pages \cite{Indexing}. The extracted (\textbf{indexed}) information is quality-checked and then published on the organization's website FamilySearch.org where users can search for information about their ancestors.
In 2013, over 1 billion data posts for births, deaths and marriages had been indexed by volunteers \cite{Billion}.

The number of photographed pages grow faster than what the available pool of volunteers can process. Thus, there is a need for automating as much of the indexing work as possible. However, indexing is difficult to automate because (1) the records are written in different languages, (2) they come from a vast array of locations and periods in time and (3) there is considerable variation in both spelling and page structure.
The handwriting style differs from scribe to scribe although it may be somewhat similar during the same time period in the same region.
%Although the handwriting style may be similar during the same time period, individual scribes write differently.
% there is considerable variance between different scribes.

If the images were correctly transcribed by a \textbf{handwritten text recognition} (HTR) system, it would still take considerable natural language processing to extract the information. Especially since there is little training data for languages over different time periods.

An easier problem to solve than extracting all relevant information in the images is grouping the images by year, record type and location. This task is called \textbf{waypointing} \cite{Waypointing} and it currently requires considerable manual labor by genealogy experts.

Annotating each image with year, record type and location would solve waypointing as well as constituting a stepping stone towards fully or partially automated indexing.
Image annotation is an easier problem than handwriting recognition because it only needs to consider information which belongs to a small domain compared to the full vocabulary of a natural language and all possible names.
