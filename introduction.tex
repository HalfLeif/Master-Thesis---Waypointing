
\section{Introduction}

FamilySearch is a large non-profit genealogy organization \cite{FamilySearchAbout}. They have been photographing handwritten historical population records in many countries for many years.

In 2006, the \textbf{Indexing} project started where volunteers manually extract relevant information from the photographed pages \cite{Indexing}. The extracted (\textbf{indexed}) information is quality-checked and then published on the organization's website FamilySearch.org where users can search for information about their ancestors.
In 2013, over 1 billion data posts for births, deaths and marriages had been indexed by volunteers \cite{Billion}.

The number of photographed pages grow faster than what the available pool of volunteers can process. Thus, there is a need for automating as much of the indexing work as possible. However, indexing is difficult to automate because (1) the records are written in different languages, (2) they come from a vast array of locations and periods in time and (3) there is considerable variation in both spelling and page structure.
The handwriting style differs from scribe to scribe although it may be somewhat similar during the same time period in the same region.
%Although the handwriting style may be similar during the same time period, individual scribes write differently.
% there is considerable variance between different scribes.

If the images were correctly transcribed by a \textbf{handwritten text recognition} (HTR) system, it would still take considerable natural language processing to extract the information. Especially since there is little training data for languages over different time periods.

An easier problem to solve than extracting all relevant information in the images is grouping the images by year, record type and location. This task is called \textbf{waypointing} \cite{Waypointing} and it currently requires considerable manual labor by genealogy experts.

Annotating each image with year, record type and location would solve waypointing as well as constituting a stepping stone towards fully or partially automated indexing.
Image annotation is an easier problem than handwriting recognition because it only needs to consider information which belongs to a small domain compared to the full vocabulary of a natural language and all possible names.


\section{Genealogy}

Genealogy is the study of one's ancestors. It involves finding who they were, the time and place of their births, where they lived, how their lives were, how their lives eventually ended and how their families continued to present day.

Getting to know and comprehend the lives of one's ancestors can have great personal significance as one can feel more connected with people in centuries past. It may increase the appreciation for parents and forefathers and their efforts so that oneself could be born.
Experiences can also be learned from important events, stories and hardships in the lives of one's ancestors and
%one can more easily appreciate
it becomes more apparent
how many things are different today, although as human beings we are perhaps not so different.

Many people are familiar with the lives of their parents and perhaps also the their grandparents from just listening to them.
However, each generation further back requires more study and effort to explore as the records get more sparse, get harder to access, are written in an older language and in a handwriting style that get increasingly different from the handwriting of today. Furthermore, because each child has two biological parents, the number of ancestors increases at least by a factor of two in each generation.

The advance of technology has been very instrumental in making the genealogical research easier and more accessible. Instead of traveling to different vaults of the records, people could order microfilms with photographs to view at home or at a nearby genealogy center. With the Internet, people can currently access many records directly online as images. A future big leap forward consists of the indexing effort which makes it possible for people to search the records with search engines and databases instead of manually scanning through and reading the pages.

Manually indexing a book from cover to cover has a very high throughput of data compared to searching the pages for a single entry because (1) it takes time to get accustomed to the handwriting of that particular scribe as well as locations in that parish and (2) the time for each entry is significantly reduced as one does not need to search through many pages. So, although it is not viable for a single person to do, if many people contribute, it lowers the total workload in the end for the community.

However, indexing takes a lot of effort and also some experience in reading old handwriting. If computers could index the records themselves with only minor human contribution, that would significantly improve the indexing rate and hence produce a very powerful tool for the entire genealogical community.

Solving waypointing is not as significant as indexing but would constitute a stepping stone towards indexing by extracting information from the records in an automated manner.
% as well as reducing manual labor for waypointing.
Waypointing in itself is also useful as more images can be published online under the correct categories. Automating waypointing would free time in favor of genealogical research and indexing.

\section{Goals}

The goal of the thesis is to make an open-source prototype that automatically extracts high-level information for each photographed page of historical population records. The information under consideration is 1. year, 2. page number and 3. record type, where the record type is one of \{blank, birth, death, marriage\}.
% If time permits, extracting location (such as parish) will also be considered.

The system should learn from the currently indexed and waypointed data in a fully autonomous manner without explicit feature engineering.
% However, standard pre-processing techniques such as median filtering and gray level normalization may be necessary.

%A perfect classification is unrealistic so the resulting groupings of images will need some manual inspection before publication. However, it should be sufficient to check some of the pages in a group instead of going through every page. Thus the manual workload should be significantly decreased. If the precision of the system is too low, correcting errors would be equally troublesome as doing the waypointing manually. Thus, the precision of the classification should be as high as possible.

% While maximizing the precision, the system should have a reasonable throughput. Processing millions of photographed pages should be viable.

\section{Delimitation}

Although FamilySearch has indexed images in many languages from many places,
%Although there are images in many languages from many places,
the thesis only considers images of handwritten text in a single language from a single country but over a longer period of time, possibly up to three hundred years.
However, the techniques applied should be general enough to handle many different languages.
%, at least for extracting year and page number.

% We primarily consider deep learning techniques for solving the problem in contrast to conventional methods of handwriting recognition.

In order to simplify the model,
the system will classify each page with a single year although there may be none or several years present. For example, one photograph can contain baptisms from 1793 to 1797. In order to solve indexing, it is however necessary to correctly recognize all present values.


\subsection{Related work}
...

