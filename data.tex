\section{Datasets}
% TODO add some pictures here?
% Describe data sets MNIST, IRIS, SWE as well as other unused ones.

% TODO add citations for each mentioned data set?

\subsection{Datasets for handwriting recognition}

Several collections of historical documents have been line segmented and transcribed for public use \cite{esposalles}: George Washington, Parzival, Saint Gall, RODRIGO and GERMANA. However, they consist of prose written by one or two scribes and not civil population records which are typically written by many different scribes.

The Esposalles dataset contains Marriage licences from 202 pages \cite{esposalles}. Experts have manually segmented and transcribed each word.

% This dataset is useful for segmentation techniques

% The Esposalles dataset was created to accelerate research in information extraction from population records \cite{esposalles}.

% The Esposalles data set consists of manually segmented and transcribed words from 202 pages of Spanish marriage licenses from the Cathedral of Barcelona \cite{esposalles}. Making this dataset is naturally a very slow process since it requires


% Although there are many thousand pages of marriage records in the Archives of the Cathedral of Barcelone, only 202 of them have been segmented and transcribed as part of the Esposalles data set.

\subsection{IRIS}

While the Esposalles dataset focus on individual words transcribed by experts, the IRIS dataset contain record information extracted by volunteers in FamilySearch's indexing program \cite{Iris}. Thus the transcriptions are not a complete representation of the text but of the semantic content.

The IRIS dataset consist of four collections of population records totaling nearly $50000$ images. Two collections are the 1930 population census in the US and Mexico. These images consist of big tables with names and occupations in the cells. Another collection contain marriage licenses between 1837-1944 in Arkansas, these pages consist of printed license templates with handwritten text in the blanks. The last collection contains more than 10000 pages of free form written records from French parishes between 1533-1906.

\subsection{Swedish population records}

\begin{table}
\centering
\begin{tabular}{ l | r r r}
Name & Collection & Total images & Indexed images \\
\hline
Örebro	& 1647578	& 638873\footnote{\url{https://familysearch.org/search/collection/1647578}}	& 14330 \\
Uppsala	& 1647598	& 693757\footnote{\url{https://familysearch.org/search/collection/1647598}}	& 8869 \\
Södermanland	& 1647693	& 617735\footnote{\url{https://familysearch.org/search/collection/1647693}}	& 11827 \\
Kalmar	& 1930243	& 617325\footnote{\url{https://familysearch.org/search/collection/1930243}}	& 2081 \\
Jönköping	& 1930273	& 721027\footnote{\url{https://familysearch.org/search/collection/1930273}}	& 3637 \\
Västernorrland	& 1949331	& 273798\footnote{\url{https://familysearch.org/search/collection/1949331}}	& 2324 \\
\hline
Sum & & 3562515 & 43068
\end{tabular}
\caption{Collections of indexed Swedish records from FamilySearch.}
\label{tab:collections}
\end{table}

In cooperation with FamilySearch for making this thesis, we have been granted access to manually indexed images from six Swedish collections, see Table \ref{tab:collections}. They contain records of births, deaths and marriages between 1627 and 1890. Like with the IRIS dataset, the images are not completely transcribed but the relevant genealogical information is extracted.
